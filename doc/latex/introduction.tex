\section{Introduction and Motivations}
Most of the major automotive manufacturers estimate that they will release an autonomous (self-driving) car within the next ten years. Indeed, autonomous vehicles promise to be capable of both safer and smoother traffic than human drivers can currently achieve. Computers will not only steer the vehicles, but most likely also organize traffic in a bigger picture. As self-driving cars will be introduced to our streets, we will inevitably have to deal with a situation where both human and robotic drivers share the road. For computers to be able to predict and evaluate human driving, we require computer models that accurately reproduce human driving behaviour. In this sense, the understanding of traffic dynamics is as relevant as ever.

Available models (e.g. the \emph{Intelligent Driver Model}, see section \ref{sec:model}) are able to predict the transition from smooth, homogeneous traffic to stop-and-go waves, as, for example, the density of the vehicles exceeds a certain threshold. It is well established \cite{treiber1999, treiber2000, treiber2013} that the instability may be triggered by a short, localized increase in vehicle density. This cluster scatters at an obstacle (i.e. a region with e.g. a lower speed limit), and create a traffic jam much larger in size than the original perturbation.

Instabilities may also form spontaneously on roads that are populated near (but below) their saturation capacity. It is this situation that we will study further. Indeed, we report that one particular parameter in the \emph{Intelligent Driver Model} has a crucial impact on the formation of these spontaneous instabilities. To the best of our knowledge no previous studies on this matter have been performed.

