\documentclass[11pt]{article} 
\usepackage[a4paper]{geometry}

\newcommand{\noun}[1]{#1}%{\textsc{#1}}
\newcommand{\HRule}{\rule{\linewidth}{0.5mm}}

\usepackage[english]{babel}
\usepackage[pdftex]{graphicx}                  % This is needed for including figures and graphics
\usepackage{amssymb}
\usepackage{amsmath}
\usepackage{wrapfig}
\usepackage[version=3]{mhchem}
\usepackage[hidelinks]{hyperref}
\usepackage{siunitx}
\usepackage{gensymb}				% for the ° symbol (\degree)
\usepackage{wasysym}
%\usepackage{todonotes}
\usepackage[toc,page]{appendix}
\usepackage{booktabs}
\usepackage{tikz}
\usepackage{subfigure}
\usepackage{placeins}
\usepackage{caption}
\usepackage{pdfpages}
\usepackage{graphicx}
\usepackage{natbib}



\begin{document}

\begin{titlepage} \begin{center}
        
    % Upper part of the page. The '~' is needed because \\ % only works if a paragraph has started.
    \begin{flushright}
        \includegraphics[width=0.4\textwidth]{ETHlogo.pdf}~\\[1cm]
    \end{flushright}
        
        
    \vspace{4cm}
    \LARGE{ 	Lecture with Computer Exercises:\\ }
    \LARGE{ Modelling and Simulating Social Systems with MATLAB\\}
    % Pol: MATLAB?! Seriously?
    
    \bigskip\bigskip
    
    \small{Project Report}\\
    
            
    % Title
    \HRule \\[0.4cm] { \huge \bfseries New phase transition in the Intelligent Driver Model\\[0.4cm] }
    
    \HRule \\[1.5cm]
    
    % Author and supervisor 
    \noindent 
    \begin{minipage}[t]{0.4\textwidth} 
        \begin{flushleft} 
            \large \emph{Authors:}\\ Carlo \textsc{Del Don}\\ Pol \textsc{Welter} 
        \end{flushleft} 
    \end{minipage}
    \begin{minipage}[t]{0.4\textwidth}
        \begin{flushright}
            \large \emph{Supervisors:} \\
            Lloyd \textsc{Sanders} \\
            Olivia \textsc{Woolley} \\
        \end{flushright}
    \end{minipage}
    
    \vfill
    
    % Bottom of the page
    {\large \today}
        
\end{center} \end{titlepage}
\begin{abstract}
We report a new phase transition in the Intelligent Driver Model. By tuning the exponent of the interaction term, the simulated traffic flow can be rendered stable across a wide range of vehicle densities, accelerations, and decelerations.
\end{abstract}
\newpage


\newpage
\section*{Agreement for free-download}
\bigskip\bigskip
\large We hereby agree to make our source code for this project freely available for download from the web pages of the SOMS chair. Furthermore, we assure that all source code is written by ourselves and is not violating any copyright restrictions.

\begin{minipage}[t][2cm][b]{0.45\textwidth}
    \centering Carlo Del Don
\end{minipage}
\begin{minipage}[t][2cm][b]{0.45\textwidth}
    \centering Pol Welter
\end{minipage}

\vspace{5cm}
\section*{Individual contributions}
\bigskip\bigskip
The authors have contributed equally to this work.

\newpage

\tableofcontents

\newpage

\section{Introduction and Motivations}
Most of the major automotive manufacturers estimate that they will release an autonomous (self-driving) car within the next ten years. Indeed, autonomous vehicles promise to be capable of both safer and smoother traffic than human drivers can currently achieve. Computers will not only steer the vehicles, but most likely also organize traffic in a bigger picture. As self-driving cars will be introduced to our streets, we will inevitably have to deal with a situation where both human and robotic drivers share the road. For computers to be able to predict and evaluate human driving, we require computer models that accurately reproduce human driving behaviour. In this sense, the understanding of traffic dynamics is as relevant as ever.

Available models (e.g. the \emph{Intelligent Driver Model}, see section \ref{sec:model}) are able to predict the transition from smooth, homogeneous traffic to stop-and-go waves, as, for example, the density of the vehicles exceeds a certain threshold. It is well established \cite{treiber1999, treiber2000, treiber2013} that the instability may be triggered by a short, localized increase in vehicle density. This cluster scatters at an obstacle (i.e. a region with e.g. a lower speed limit), and create a traffic jam much larger in size than the original perturbation.

Instabilities may also form spontaneously on roads that are populated near (but below) their saturation capacity. It is this situation that we will study further. Indeed, we report that one particular parameter in the \emph{Intelligent Driver Model} has a crucial impact on the formation of these spontaneous instabilities. To the best of our knowledge no previous studies on this matter have been performed.


\section{Description of the Model}
We have implemented the \emph{Intelligent Driver Model} (IDM). The IDM, first introduced by Treiber et al. \cite{treiber1999, treiber2000}, is deterministic and continuous (both in time and space). It describes the movement of each vehicle by a set of differential equations, giving the vehicle's acceleration as a function of its current velocity $v_\alpha$, the gap to the leading vehicle $s_\alpha$, etc. 

Formally the acceleration $\dot v_\alpha$ of vehicle $\alpha$ reads 
\begin{equation}
\dot v_\alpha = a\left(\underbrace{1-\left(\frac{v_\alpha}{v_0}\right)^\delta}_{\text{free road term}} - \underbrace{\left(\frac{s^*(v_\alpha, \Delta v_\alpha)}{s_\alpha}\right)^\gamma}_{\text{inteaction term}}\right).
\label{eq:IDM}
\end{equation}


This equation contains two parts, the \emph{free road term}, and the \emph{interaction term}. For a low vehicle density (i.e. $s_\alpha \rightarrow 0$) the above equation can be simplified to
\begin{equation}
\dot v_\alpha = a\left(1-\left(\frac{v_\alpha}{v_0}\right)^\delta\right).
\end{equation}
For any $\delta>0$ this will result a relaxation of the velocity $v_\alpha$ to the speed limit $v_0$. The value of the exponent $\delta$ is typically chosen in between 1 (exponential relaxation) and 6 (the limit $\delta\rightarrow \infty$ corresponds to a constant acceleration). Throughout literature the most commonly used value is $\delta=4$. This is the value that has also been adopted in this work.

The \emph{interaction term} is designed to limit the cars' velocity on a densely populated road. It acts as a `repulsive potential', which tries to enforce the \emph{desired gap} $s^*$ as the distance between two cars. The desired gap is calculated as
\begin{equation}
s^*(v_\alpha, \Delta v_\alpha) = s_0 + \max\left(v_\alpha T + \frac{v_\alpha \Delta v_\alpha}{2\sqrt{ab}},\;0\right).
\label{eq:desired_gap}
\end{equation}
It contains
\begin{itemize}
    \item a minimum distance $s_0$, which is respected even at stand still,
    \item a safety distance $v_\alpha T$ based on the drivers' desired time headway\footnote{The time headway $T$ is the time between the moment the first vehicle's bumper passes a stationary point, and the moment the next vehicle passes it.} $T$,
    \item a breaking term $\frac{v_\alpha \Delta v_\alpha}{2\sqrt{ab}}$, with the `comfortable' deceleration $b$.
\end{itemize}
The breaking term inhibits large closing speeds between vehicles. It has been designed in such a way that for $\gamma = 2$, the deceleration does not exceed $b$ except in emergency situations (for more information on the asymtpotic behaviour of this term, see \cite{treiber1999}). In the present work, particular focus has been paid precisely to  $\gamma \ne 2$.

Indeed, while for $\gamma=2$, the interaction term in eq. (\ref{eq:IDM}) represents a Coulomb like force field. For $\gamma>2$, the repulsive walls of the potential will become steeper. In the following we will demonstrate that increasing $\gamma$, while leaving the other parameters fixed, will suppress any instabilities in the traffic flow.

\subsection{Boundary conditions}
As for the boundary conditions, there are two obvious choices: either add and remove vehicles during the simulation run (as is done e.g. in \cite{treiber1999}, or make the road periodic, i.e. ring like \cite{treiber2015}. It is this second variant that has been implemented for this project. It makes the model significantly simpler, as apart from a transient time at the beginning of the simulation, no border effects appear. On the other hand, it provides limited ability to change the amount of cars on the road at run-time. In \cite{treiber1999, treiber2006} for instance, the instability is triggered by a short increase of the influx.

\subsection{Limits of the model}
The differential equations (\ref{eq:IDM}) are integrated via the following Runge-Kutta scheme:
\begin{align*}
v_\alpha(t+\Delta t) &= \dot v_\alpha(t)\Delta t+v_\alpha(t) \\
x_\alpha(t+\Delta t) &= \frac{1}{2}\dot v_\alpha(t) (\Delta t)^2 + v_\alpha(t)\Delta t + x_\alpha(t)
\end{align*}
For the case where $\dot v_\alpha$ does only depend on $x_\alpha$, but not on $v_\alpha$, this scheme is known as the Verlet method, and converges with quadratic order. In the present case, these requirements are not met, so second order convergence might not actually be achieved. No tests on this have been carried out, as even first order methods have been proven to be accurate enough \cite{treiber2015}.

Note that the IDM does not model the finite reaction time of an actual human driver\footnote{The finite time step $\Delta t$ is \emph{not} to be mistaken for a reaction time.}. Indeed, when a realistic reaction time is included, IDM-like simulations show very poor stability when compared to empiric human driving . This led to the conclusion that in reality drivers do look ahead further than only the nearest neighbour \cite{treiber2006}.

The road on which the vehicles travel is assumed to be single lane (no overtaking) and straight (constant speed limit). All drivers and vehicles are assumed to be identical.

\section{Implementation}
For performance reasons, the simulation engine has been implemented as a \textsc{C++} program. It reads the simulation parameters from a text file, runs the simulation, and saves the desired values (e.g. vehicle position \& velocity as a function of time) in another file.

This back-end is controlled by a suite of tools written in \textsc{Python}, which also perform data analysis. The \textsc{Python} front-end feeds parameters to the engine, and processes its output.

\section{Simulation Results and Discussion}

\section{Summary and Outlook}

\bibliographystyle{plain}
\addcontentsline{toc}{section}{References}
\bibliography{reference}



\includepdf{declaration_originality}



\end{document}  



 
